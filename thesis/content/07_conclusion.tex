\chapter{Conclusions}
\label{cha:conclusion}

The problem of unpaid content is something that already existed from the early days of the World Wide Web. As of today, the main revenue model that is applied at online content is (targeted) web advertising.

% summery

This thesis presents a fully working prototype of system that could be an alternative to web advertising. The idea of the system is that web advertising generates a very small amount of revenue for each impression or click. In order to offer an advertisement free web experience that is still fair to the publisher, the system pays the revenue that would be missed out on. 

The challenges of building such a system are very interesting. Firstly, as the revenue of a single advertisement impression is a matter of tenths of cents, there is a system needed that can facilitate payments with these very low amounts and even lower transaction fees. Secondly, there needs to be a model that determines what amount gets paid to a publisher, so that it is fair to both the internet user and the publisher. Thirdly, in order to make such a system successful, wide adoption is needed. Therefore, it should be easy to take part in the system from both sides.

The prototype that is featured in this thesis is build on the Lightning Network. This network works as an extra layer on a blockchain and can facilitate very fast and cheap payments with low fees. The prototype runs outside the browser and functions as a wallet that supports the Lightning Network. The publisher implements a JavaScript library that is able to communicate with the application that runs outside the browser and can request a small payment. Several security measures are in place in order to prevent fraudulent parties to empty your wallet. For these security measures, the existing infrastructure, such as DNS records and SSL certificates, are applied. 

% what is not done
The scope of this research is limited on a couple of aspects. Firstly, the support of the prototype is limited to the desktop computer. However, the application is developed cross-platform, but without support for mobile. Secondly, this thesis is not about revenue models for online content. The revenue model applied in this prototype is pretty straight forwarded might not work as intended in some cases. 

% future work, interesting to research
 

\section{Recommendations and improvements}
It has been proved that the concept works in a real life scenario. However, there is still plenty of room for improvement. As stated in section \ref{sec:electron}, the user experience might be improved a lot when a light wallet, such as Neutrino is properly implemented. This takes away the times it takes to download the entire blockchain. Also, the lightning-app, that is featured in that section introduces the autopilot feature of the Lightning Network so that the user does not have to understand the concept of lightning channels. 

Another aspect of the system that requires some research is the revenue distribution model. Right now, this is based on a maximum distribution per hour, which means that visiting more websites per hour will result in a smaller revenue per website. It is very hard to take the value of content into account. One approach that has been experimented with, is to make the revenue share dependent on the Pagerank score. However, this also does not say anything about the value of a certain page. Another approach could be to let one trusted authority determine the value of a Wikipedia article and the value of a photo on Instagram. Anyway, It would be very interesting to have other views on this matter. 