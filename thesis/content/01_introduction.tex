\chapter{Introduction}
\label{cha:introduction}

The business of online advertising has evolved into a landscape which is not transparent anymore. A handful of large advertisement firms are controlling practically every online ad you see. Almost every movement during the visit of a regular website is sent in an obfuscated way to the advertisement broker, without any visible sign to the visitor. This makes the whole browsing experience obnoxious, especially now it turns out that entire societies are being influenced by the power of advertisement networks, as we have seen in the Cambridge Analytica scandal \cite{FakeWebPage11}.

Several publishers have been experimenting with alternative ways of generating income. Currently, some of them are selling subscriptions, asking for donations or using the visitors' computer for cryptomining \cite{ruth2018digging}. But these models do not seem to be a real substitute for advertisement networks. 

In this master thesis, which is written at the Service-centric networking research group at the Technische Universität Berlin, the main focus lies at solving this so called unpaid content problem while assuring the privacy of the user and keeping the costs low. The increasing possibilties in the field of blockchain technology are of great use for such a solution and therefore also a key building block of the proof of concept.

The concept, as discribed in chapter \ref{cha:conceptanddesign}, features a system that runs in the background while browsing the web. If the users visits a publisher that also supports the system, a message will be shown to the user indicating that it is possible to hide the advertisements and pay a small amount per pageview instead. When this permission is granted, the user will not see any advertisements on that particular website again, but contribute by sending small payments to the publisher instead.

As this research is made possible by public money, the entire process is kept as transparent as possible. This is achieved by publishing everything related to this thesis under a permissive free software license on Github \footnote{\url{https://github.com/lightning-sprinkle}}.

\section{Research statement}

This reseach will investigate the possiblities of new technologies in order to solve the unpaid content problem. The following research question is defined:
\vspace{1em}

\textit{How can the unpaid content problem be solved in a cheap, privacy preserving and transparent way?}
\vspace{1em}

\noindent This research question is split up in the following subquestions:

\begin{itemize}
  \item What current revenue models are used in order to solve the unpaid content problem?
  \item How is privacy preserved in the current models?
  \item What are the costs of the current models?
  \item What is the amount of transparency in the current models?
  \item What are the conditions, that an alternative model should adopt in order to be at least comparable to existing models?
  \item How to realize and implement a comparable revenue model that follows these conditions?
\end{itemize}

\section{Methodology}
\label{sec:methodology}
In order to explain the different models and concept, a couple of roles will be used througout this thesis.

\begin{description}
  \item[Unpaid content] \hfill \\ Content that is freely available on the internet (without a subscription or payment), such as news articles and video's.
  \item[Publisher] \hfill \\ The owner of the website that provides the unpaid content
  \item[User] \hfill \\ The visitor of the website that consumes the unpaid content
  \item[Ad broker] \hfill \\ A third party providing advertisements to the user in order to generate revenue for the publisher
\end{description} 