\chapter{Societal Impact}
\label{cha:societalimpact}

Online advertising is an important pillar of the World Wide Web, around 36.6\% of all domains is using an online advertising network\footnote{\url{https://w3techs.com/technologies/overview/advertising}}, which means that practically every internet user is exposed to online advertising. With a worldwide internet penetration of 59.6\%\footnote{\url{https://www.internetworldstats.com/stats.htm}}, this exposure to online advertising accounts for the majority of the world population. Therefore, it is important to discuss the potential societal impact of the concept of this thesis.

\section{Relevance}
Online advertising and targeted advertising especially, is quite often a subject of debate. Firstly, advertisements are having a negative impact on the browsing experience. This becomes clear if the engagement of internet users with and without ad blocker is compared: \textit{Users who installed an ad blocker were active in the browser for around 28\% more time on average, and loaded 15\%more pages (URIs), controlling for baseline activity}\cite{googlefee}.

Secondly, advertisements are consuming a lot of bandwidth and battery, especially on mobile devices. A study finds that in mobile games, 4-15\% of the consumed energy is spent on\\ advertisements\cite{prochkova2012energy}. In some countries, mobile data plans are still very expensive. As advertisements are taking up a lot of bandwidth, this even results in extra costs. Depending on the data plan, advertisements might account for around 4.5 to 7 USD of the price of the mobile data plan\cite{van2012costs}.

Thirdly, advertising networks are also used for criminal activities. Malvertising is the use of online advertising as a vector to deliver malware. It involves the injection of malicious or malware laden advertisements into legitimate and recognized websites\cite{dwyer2016gone}. As of 2020, advertisement networks have limited the possibilities to exploit these advertisements. However, for the advertisement networks, it is still not possible to control what happens after the user clicks on the advertisement.

Fourthly, targeted advertising is also used as a method to manipulate the public opinion and even used as a political tool. This happened for example during the Trump campaign and Brexit refendum\cite{cadwalladr2018cambridge}.

Therefore, online advertising might be considered as a necessary evil on the World Wide Web, mainly due to a lack of proper alternatives.

\section{Feasibility}
With such a system, there needs to be a high adoption rate in order to be successful. If only a hand full of publisher supports the system, it will never have any impact. The good news is that supporting the system does not have any drawbacks. If for example, Wikipedia, decides to implement the Lightning Sprinkle Publisher Library, users without the Lightning Sprinkle User Service will not notice anything different. Therefore, adopting the system as a publisher will only have positive effects.

For the internet user, however, a lot of individual effort is requested. Firstly, the internet user needs to install the Lightning Sprinkle User Service. Secondly, the internet user has to transfer a cryptocurrency to the wallet of the application, which is a lot to ask compared to just installing an ad blocker. 
But most important: it costs money. The unpaid content problem is mostly a problem at the side of the publisher, not at the side of the user. Therefore, there is almost no intrinsic motivation to use such a system as long as users are not complaining about the current state of targeting and tracking. 

\section{Suggestions}
Based on the concerns about the feasibility described above, a couple of suggestions is made. The problem seen from a societal perspective is mostly about the amount of effort that it takes and the costs that are involved. The effort that is required can be reduced by introducing a hybrid solution: a solution that works in the same decentral way as the concept presented in this thesis, but also supports a central commercial implementation that does not require installing extra software and the use of a cryptocurrency. The cost aspect is harder to solve: as described in the related work, free is a stable strategy and therefore hard to compete with. One approach that comes from the airline industry can be applied: it is better to sell a seat for one dollar than to leave it empty. A pay what you wish model with a minimum of 0 can be applied. However, these pricing strategies are outside the scope of this thesis. Another approach to make it more feasible is to involve the government in this matter and educate the internet users about the risks and cons of targeted advertising. Such a campaign might increase the use of alternatives like the concept proposed in this thesis.


\section{Changes for internet users}
If this system becomes a success, in might also be built in into browsers by default. The following impacts are based on a scenario where the system works and is widely adopted. 

\subsection{Positive impact}
The positive impact for internet users can be seen on an economical level and from an experience point of view. Firstly, advertisements are modifying consumer behavior. They try to impact the to-buy or not-to-buy decision\cite{johnson2007consumer}. A study found that a ban on advertising does not affect consumption as a whole, but only affects the market shares of individual brands\cite{advertisementsconsumption}. Therefore, an alternative to targeted advertising will not reduce the consumption behavior of the internet user. However, the user might choose for a particular product based on a more objective decision, that is less influenced by marketing.

\subsection{Negative impact}
On the negative side, alternatives to targeted advertising are mostly not free. As seen in the related work section, paid content works best with a high educated, white male audience. Therefore, a non-free alternative to targeted advertising might increase the cap between poor and rich. Ad free browsing might even become a privilege for a selected group of people.
