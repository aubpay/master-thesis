\chapter{Societal Impact}
\label{cha:societalimpact}

Online advertising is an important pillar of the World Wide Web, around 36.6\% of all domains is using an online advertising network\footnote{\url{https://w3techs.com/technologies/overview/advertising}}, which means that practically every internet user is exposed to online advertising. With a worldwide internet penetration of 59.6\%\footnote{\url{https://www.internetworldstats.com/stats.htm}}, this accounts to the majority of the world population. Therefore, it is important discuss what societal impact the concept of this thesis can make.

\section{Relevance}
Online advertising, especially targeted advertising is quite often a subject of debate. Firstly, advertisements are having a negative impact on the browsing experience. This becomes visible if the engagement of internet users with and without ad blocker is compared: \textit{Users who installed an ad blocker were active in the browser for around 28\% more time on average, and loaded 15\%more pages (URIs), controlling for baseline activity}\cite{googlefee}.

Secondly, advertisements consuming a lot of bandwith and battery, especially on mobile devices. A study finds that in mobile games, 4-15\% of the consumed energy is spent on\\ advertisements\cite{prochkova2012energy}. In some countries, mobile data plans are still very expensive. As advertisements are taking up a lot of the bandwidth. Depending on the data plan, advertisements might account for raound 4.5 to 7 USD of the price of the mobile data plan\cite{van2012costs}.

Thirdly, advertising networks are also used for criminal activities. Malvertising is the use of online advertising as a vector to deliver malware. It involves the injection of malicious or malware laden advertisements into legitimate and recognized websites\cite{dwyer2016gone}. As of 2020, advertisement networks have limited the possibilities to exploit these advertisements. However, for the advertisement networks, it is still not possible to control what happens after the user clicks on the advertisement.

Fourthly, targeted advertising is also used as a method to manipulate the public opinion and even used as a political tool. This happened for example during the Trump campaign and Brexit refendum\cite{cadwalladr2018cambridge}.

Therefore, online advertising might be considered as a necessary evil on the World Wide Web, because of a lack of proper alternatives.

\section{Feasibility}
With such a system, there needs to be a high adoption rate in order to be successful. If only a hand full of publisher supports the system, it will never have any impact. The good news is that supporting the system does not have any drawbacks. If for example, the Guardian, decides to implement the Lightning Sprinkle Publisher Library, users without the Lightning Sprinkle User Service will not notice anything different. Therefore, adopting the system as a publisher will only have positive effects.

%geld user


\section{Suggestions}
In order to make such a system successful, it might be an idea to 


\section{Changes for internet users}
If this system becomes a success, in might also be built in into browsers by default. 

\subsection{Positive impact}
The positive impact for internet users can be seen on an economical level and from an experience point of view. Firstly, advertisements are modifying consumer behavior. They try to impact the to-buy or not-to-buy decision\cite{johnson2007consumer}. A study found that a ban on advertising does not affect consumption as a whole, but only affects the market shares of individual brands\cite{advertisementsconsumption}. Therefore, an alternative to targeted advertising will not reduce the consumption behavior of the internet user. However, the user choose for a particular product based on a more objective decision.

\subsection{Negative impact}

