\chapter{Performance Analysis \& Evaluation}
\label{cha:evaluation}

The concept that is designed in this thesis is evaluated with two different software projects: the Lightning Sprinkle User Service and the Lightning Sprinkle Publisher Library, which are written in Python and Javascript respectively. 

\section{Technical}

The goal of these software projects is to prove that it is possible to implement that is an answer to the concept that is stated in section \ref{sec:uaps}. These requirements will be used as part of the evaluation, combined with a couple of performance tests. Performance tests that include: the scalability of the platform and response time. 

\noindent The original concept from section \ref{sec:uaps} is:

\vspace{1em}

Concept: \textit{Distribute a small amount of money over the publishers behind the websites you visit and hide the advertisements}

\vspace{1em}

\noindent The challenges from the same concpet can be translated into the following requirements:
\begin{enumerate}
  \item Minimal configuration
  \item Decentral
  \item Fair distribution
  \item Tamper proof
\end{enumerate}

Firstly, the minimal configuration requirement can be observed from two perspectives: a world where cryptocurrencies are widely adopted, and the real world. In the first case, the system is just a matter of downloading the application and depositing some money on the bitcoin wallet that comes with the application. After the deposit is confirmed on the blockchain, the rest will go automatically: The application will configure the lightning wallet by opening a channel and will accept requests from publisher that are using the Lightning Sprinkle Publisher Library. 

In the real world, it is not so easy to obtain any cryptocurrency. In most countries, there are laws in order to prevent money laundering and backing of terrorism. For example, in the Netherlands, there is an Anti-Money Laundering and Anti-Terrorist Financing Act\footnote{\url{https://www.toezicht.dnb.nl/en/4/6/51-204766.jsp}}, that requires that any financial institution gathers data about their customers, which also includes cryptocurrency exchanges. The process of converting any fiat currency to a cryptocurrency might consist out of registering somewhere and uploading a copy of an identity document. This makes the whole application more cumbersome and hard to use by non-tech-savvy users. Therefore, the first requirement is met partly because of external limitations.

Secondly, with the decentral requirement, there should not be a single authority that has any power over the network as a whole. This requirement is almost met completely. The system relies on the bitcoin blockchain, which is decentral by design. This also accounts for the lightning network, as this network is also decentral by nature. The only central authorities that the system relies on, are the certificate authorities. A certificate authority might revoke a certificate, so that the Lightning Sprinkle User Service does not send automated payments anymore. However, revoking a certificate results in side effects that are much worse: the browser will reject to connect to the website in first place. 

Thirdly, the fair distribution requirement is hard to satisfy. In first place, there is no existing distribution model that handles this problem. For example, the Brave browser just distributes the money over the publishers and does not take into account things like the frequency of visits are the credibility of the website. This thesis tries to come up with a model that takes frequency of visits into account, but it is still hard to make it really fair. Is a news website that someone visits ten times per day also ten times as valuable as one single Wikipedia article? This topic still has plenty of room for discussion. 

Fourthly, the security of the system in terms of fraud is moderate. The attack where a fraudulent party is able to register a bunch of domain names that all try to request a payment is solved by only allowing domain names with an OV or EV certificate to request a payment immediately. Of course, this is not bullet proof as a party with enough budget is still able to register enough domain names with these expensive certificates. Other attacks might come from malware that is installed on the computer, however, this is in most practical applications of crytocurrencies the case.

\section{Relevance}
