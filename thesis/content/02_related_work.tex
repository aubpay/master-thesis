\chapter{Related Work}
\label{cha:relatedwork}

Revenue models that are applicable on the internet in order to monetize online content is a topic that has been actively researched and experimented with over the past few decades. This chapter will dive into the related work on both a technical and an economical level. It is structure in a way so that three different research areas will be discussed. Firtsly, the economic aspects of unpaid content are reviewed. Secondly, an overview of the technical solutions to the unpaid content will be given. This will include both micropayments, subscription models and online advertising. Lastly, the possiblities that arise in the blockchain era will be discussed.

\section{Economic aspects}

Even in the early days of the World Wide Web, the problem of unpaid content existed.  These approaches include paywalls and web advertisements. For example, the \textit{Wall Street Journal} implemented already a hard paywall in 1996, which is still in place as of today, with over 2 million subscribers as of February 2020 \cite{firstpaywall}. 
Alternatives to hard paywalls are soft paywalls. The difference between both types is that soft paywalls are trying to convince potential customers to subscribe by giving them a free sample of the content. For example, the \textit{New York Times} has implemented a soft paywall with a limit of 5 free articles per month \cite{cook2012paying}.  %https://www.niemanlab.org/2020/02/the-wall-street-journal-joins-the-new-york-times-in-the-2-million-digital-subscriber-club/

% Salwen, Michael B.; Garrison, Bruce; Driscoll, Paul D. (2004). Online News and the Public. Routledge. p. 136. ISBN 978-1-135-61679-3.

Paywalls, however are fairly easy to circumvent. This especially accounts for soft paywalls. Therefore, publishers are trying to implement counter measures in order to enforce a subscription. For example, the \textit{New York Times} attacked one popular circumvention method, the use of an incognito window. With behavioral analysis, it is possible to find out that the user is using an incognito window, so that the \textit{New York Times} is able to prevent the free article from being served.   %https://www.nyulawreview.org/wp-content/uploads/2018/08/NYULawReview-90-1-Troupson.pdf

\section{Technical solutions}

In order to generate revenue from online content, there are two technical solutions broadly adopted: (micro)payments and online advertising. 

\subsection{Online advertising}
Advertising is a method to draw attention to a product, service or event in order to promote sales or attendance. Since the early days of the word wide web, this industry has also expanded to the internet. The first advertisement on the world wide web is possibly from 1994 on HotWired.com, which was bought by AT\&T and had a click through rate of 44\% \cite{firstbanner}. Meanwhile, the online advertising industry is very profitable and has evolved into the core business of the world wide web.

This section will gives an overview of the current role model of the online advertisement industrsy and take a closer look at the different approaches in online advertising and their privacy aspects. Lastly, the research field of privacy-friendly alternatives will be discussed. 

Normally, there are sereval parties involved in the advertising ecosystem. On one side, there are publishers, such as \textit{Der Spiegel} that provides online content, e.g. news articles. On the other side, there is an advertiser that provides the advertisement.

The most interesting part, however, is the ad platform. Ad platforms are entities that connect the publisher with the advertiser by providing them an interface to match demand and supply. Due to the wide range of different publishers and users that are reachable by ad networks, it becomes really efficient to allocate ad space. Ad platforms can even be concidered as the central hub in the online advertising industry \cite{estrada2017online}. When a user visits the website of a publisher, the browser communicates with the webserver. The browser receives the content that is displayed to the user. Along with this content, additional scripts that are associated with an ad network are also delivered to the browser and executed. These scripts are triggering a connection to the ad exchange. The ad exchange is able to serve extra commercial content (advertisements) over this connection, which will be embedded into the page by the script.


% approaches

% privad

\subsection{Payments}





\section{Blockchain}

\section{Brave Browser}
In 2016, Brave Software launched a browser that blocked ads and trackers by default, the Brave Browser \footnote{\url{https://www.brave.com}}. During the introduction, Brave Software also shared their plans for a Brave Publisher Ads program to pay publishers back fair a share of internet revenue. As of 2020, their service is called "Brave rewards program", and any content creator can enroll in order to get paid for content. 

\subsection{Basic Attention Token} 
In order to achieve a system that makes it possible to reward content makers on the internet, Brave introduced the Basic Attention Token. This token, which works like any other cryptocurrency, represents user attention. Their goal with this token is to trade "attention" just like any other commodity, like oil and coffee. This means that the this token can also be traded on a cryptocurrency exchange. 

\subsection{Anonize protocol}


\subsubsection{ICO}
% https://bravenewcoin.com/insights/basic-attention-token-price-analysis-publishers-increasing-across-all
Brave Software used an initial coin offering to introduce the new token to the market. The ICO happened in May 2017 and raised 156,000 ETH, which was worth around 35M USD at the time. The raised money is mostly used to pay for the development and other costs of the token. The development team exists out of 20 developers.

\subsubsection{Brave ledger system}

- Brave vault

\subsection{}


\section{Lightning Network}

