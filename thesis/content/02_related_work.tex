\chapter{Related Work}
\label{cha:relatedwork}

Revenue models that are applicable on the internet in order to monetize online content is a topic that has been actively researched and experimented with over the past few decades. This chapter will dive into the related work on both a technical and an economical level. It is structure in a way so that three different research areas will be discussed. Firtsly, the economic aspects of unpaid content are reviewed. Secondly, an overview of the technical solutions to the unpaid content will be given. This will include both micropayments, subscription models and online advertising. Lastly, the possiblities that arise in the blockchain era will be discussed.


\section{Technical solutions}

In order to generate revenue from online content, there are two technical solutions broadly adopted: (micro)payments and online advertising. 

\subsection{Online advertising}
Advertising is a method to draw attention to a product, service or event in order to promote sales or attendance. Since the early days of the word wide web, this industry has also expanded to the internet. The first advertisement on the world wide web is possibly from 1994 on HotWired.com, which was bought by AT\&T and had a click through rate of 44\% \cite{firstbanner}. Meanwhile, the online advertising industry is very profitable and has evolved into the core business of the world wide web.

This section will gives an overview of the current role model of the online advertisement industrsy and take a closer look at the different approaches in online advertising and their privacy aspects. Lastly, the research field of privacy-friendly alternatives will be discussed. 

Normally, there are sereval parties involved in the advertising ecosystem. On one side, there are publishers, such as \textit{Der Spiegel} that provides online content, e.g. news articles. On the other side, there is an advertiser that provides the advertisement.

The most interesting part, however, is the ad platform. Ad platforms are entities that connect the publisher with the advertiser by providing them an interface to match demand and supply. Due to the wide range of different publishers and users that are reachable by ad networks, it becomes really efficient to allocate ad space. Ad platforms can even be concidered as the central hub in the online advertising industry. When a user visits the website of a publisher, the browser communicates with the webserver. The browser receives the content that is displayed to the user. Along with this content, additional scripts that are associated with an ad network are also delivered to the browser and executed. These scripts are triggering a connection to the ad exchange. The ad platform is able to serve extra commercial content (advertisements) over this connection, which will be embedded into the page by the script. This method makes it possible for ad exchanges to partner up with huge amounts of publishers and serve an amount of users that is several orders of magnitudes higher \cite{estrada2017online}.

The ad platform itself, consist out of multiple components, that might also be run by different entities. Firstly, there is an ad network, which resells the ad space from to publisher to an advertiser. Secondly, another component on the ad platform is the ad exchange. These are auction based advertisements marketplaces where advertisers can bid on an ad space in realtime, which means that the auction takes place when the user visits the website of the publisher. Based on the profile of the user, certain advertisers might be more interested into buying the ad space and thus offering a higer price \cite{estrada2017online}. 

Thirdly, a data aggregator is an entity that which goal is to gather and aggregate data about the purchasing interest of the users. This data is used to provide insights to both the advertisers and the publishers to target their marketing decisions \cite{estrada2017online}.

% privad
\subsubsection{Privad}
The problem with the infrastructure mentioned above, is that everything can be controlled by one single entity. This single entity knows anything about all parties involved: advertisers, publishers and users. The behavior of a single user is tracked across multiple websites, which might be concidered a privacy concern. Guha et al. \cite{guha2011privad} developed Privad, which they call a practical private online advertising system. 

The model of Privad is slightly differs from the original online advertising role model. The model also includes the user, publisher and advertiser. However, in this model there is also a dealer and a broker present. One key difference compared to the traditional online advertising model is that the profiling (building a profile of the user based on interests) is done on the users' computer and not by a central data aggregator. Secondly, the ad platform is split up into two different entities: the brokers and the dealer. The broker is comparable to the traditional ad platform and matches the profile with advertisements. The request, however does not come from the users' computer immediately: there is a dealer placed in between. The dealer anonymizes every request before it is sent to the broker and makes sure that click fraud is prevented. The dealer cannot eavesdrop on the request, because the request is send in an encrypted from to the broker \cite{guha2011privad}. 

One concern with this approach is that a profile might be so detailed that the broker is able to find out an identity based on the profile. In orde to tackle this problem, Privad subscribes to a certain general profile that is shared with multiple other users. The user receives multiple advertisements and can pick locally which suits the best.

Trust, however is still a key element in this approach. There is no way to find out if the dealer is trustworthy. If the dealer and the broker are secretly run by the same entity, it is possible to exchange data and learn more about the user. 

\subsection{Payments}

The second business model as an alternative to online advertising is requiring payments in exchange for content. This section describes what different approaches there are in the field of online payments, subscription models and third parties offering these services.

\subsubsection{Paywalls}

Even in the early days of the World Wide Web, the phenomenom of content that is only visible with a subscription existed. Such a mechanism, is called a paywall. For example, the \textit{Wall Street Journal} implemented already a hard paywall in 1996, which is still in place as of today, with over 2 million subscribers as of February 2020 \cite{firstpaywall}. 
Alternatives to hard paywalls are soft paywalls. The difference between both types is that soft paywalls are trying to convince potential customers to subscribe by giving them a free sample of the content. For example, the \textit{New York Times} has implemented a soft paywall with a limit of 5 free articles per month \cite{cook2012paying}.

% Salwen, Michael B.; Garrison, Bruce; Driscoll, Paul D. (2004). Online News and the Public. Routledge. p. 136. ISBN 978-1-135-61679-3.

Paywalls, however are fairly easy to circumvent. This especially accounts for soft paywalls. Therefore, publishers are trying to implement counter measures in order to enforce a subscription. For example, the \textit{New York Times} attacked one popular circumvention method, the use of an incognito window. With behavioral analysis, it is possible to find out that the user is using an incognito window, so that the \textit{New York Times} is able to prevent the free article from being served \cite{troupson2015yes}.  

\subsubsection{Micropayments}
- Paypal

- Blendle

\subsubsection{Donation based}

- Wikipedia
- Flattr







\section{Blockchain}

\section{Automated payments}
In 2016, Brave Software launched a browser that blocked ads and trackers by default, the Brave Browser \footnote{\url{https://www.brave.com}}. During the introduction, Brave Software also shared their plans for a Brave Publisher Ads program to pay publishers back fair a share of internet revenue. As of 2020, their service is called "Brave rewards program", and any content creator can enroll in order to get paid for content. 

\subsection{Basic Attention Token} 
In order to achieve a system that makes it possible to reward content makers on the internet, Brave introduced the Basic Attention Token. This token, which works like any other cryptocurrency, represents user attention. Their goal with this token is to trade "attention" just like any other commodity, like oil and coffee. This means that the this token can also be traded on a cryptocurrency exchange. 

\subsection{Anonize protocol}


\subsubsection{ICO}
% https://bravenewcoin.com/insights/basic-attention-token-price-analysis-publishers-increasing-across-all
Brave Software used an initial coin offering to introduce the new token to the market. The ICO happened in May 2017 and raised 156,000 ETH, which was worth around 35M USD at the time. The raised money is mostly used to pay for the development and other costs of the token. The development team exists out of 20 developers.

\subsubsection{Brave ledger system}

- Brave vault

\subsection{}


\section{Lightning Network}

