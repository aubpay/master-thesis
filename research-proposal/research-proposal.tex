\documentclass[a4paper]{article}

\usepackage[backend=biber]{biblatex}
\usepackage{hyperref}
\usepackage{svg}

\bibliography{references}

\title{Research Proposal}
\author{Daan Middendorp}
\date{September 2019}

\begin{document}

\maketitle

\section{Introduction}
The business of online advertising has evolved to a landscape which is not transparent anymore. A handful of large advertisement firms are controlling practically every online ad you see. Almost every movement during the visit of a regular website is sent in an obfuscated way to the advertisement broker, without any visible sign to the end user. This makes the whole browsing experience obnoxious, especially now it turns out that entire societies are influenced by the power of advertisement networks, as we have seen in the Cambridge analytica scandal \footfullcite{cadwalladr2018cambridge}.

Several publishers have been experimenting with alternative ways of generating income. Currently publishers are selling subscriptions, asking for donations or using the visitors' computer for cryptomining. But these models do not seem to be a real substitute for advertisement networks.

\section{Problem}
Advertisements can help to promote commercial products or change a general opinion. With commercial products, a company pays the publisher in order to let them show an advertisement to an end user. The company needs to compensate for this payment, so in end effect the product will be sold with an additional charge. This means that the end user is still paying indirectly to the publisher. Which makes the whole ecosystem not really efficient from an end user point of view. So why not pay to the publisher directly?

At this moment there is no universal way of paying to all publishers that are serving advertisements right now. Also the publishers that are offering subscription models are asking much more than the amount of money that would be generated by serving ads.

\section{State of the art}
\label{stateoftheart}
There are basically two types of successful alternative business models for publishers to generate revenue: pay-per-view and a subscription. The problem with the first model is too much effort to make a small payment to a publisher when you just want to read one article. Blendle\footnote{\url{https://blendle.com}} tries to solve this by acting as a middleman where you are paying automatically a small amount when opening an article. The prices range between €0,09 and €1,99 per article of which 30\% is paid to Blendle \footfullcite{blendledutchnewsplatform2014}. Despite the high fees, the platform is still successful, so it turns out that people are willing to pay for content.

The second model, subscription based, functions like the old fashioned newspaper subscriptions.  The problem with this approach is that the user is tied to one particular publisher and cannot switch to another one. Have multiple subscriptions would add up in costs quickly. An example of such a publisher is the Correspondent\footnote{\url{https://thecorrespondent.com}}, which is also really successful and expanded to the US a couple of months ago. One difference between a traditional subscription model is that it is based on pay what you wish.

\section {Concept}
As stated in paragraph \ref{stateoftheart}, users are willing to pay for content. Unfortunately, there is no universal way of paying for content, without loosing money to a third party or without much effort. Over the last decade, a lot of research has been done in the field of decentralized payment methods using blockchain technology. This concept tries to adapt this engineering science in order to create a universal way of paying for content without any effort. 

\newpage

\begin{figure}[htbp]
	\includesvg{images/structure.svg}
	\caption{Concept}
\end{figure}



\end{document}