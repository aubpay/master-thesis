\documentclass[a4paper]{article}

\usepackage[backend=biber]{biblatex}
\usepackage{hyperref}
\usepackage{svg}

\bibliography{references}

\title{Research Proposal}
\author{Daan Middendorp}
\date{September 2019}

\begin{document}

\maketitle

\section{Introduction}
The business of online advertising has evolved to a landscape which is not transparent anymore. A handful of large advertisement firms are controlling practically every online ad you see. Almost every movement during the visit of a regular website is sent in an obfuscated way to the advertisement broker, without any visible sign to the end user. This makes the whole browsing experience obnoxious, especially now it turns out that entire societies are influenced by the power of advertisement networks, as we have seen in the Cambridge analytica scandal \footfullcite{cadwalladr2018cambridge}.

Several publishers have been experimenting with alternative ways of generating income. Currently publishers are selling subscriptions, asking for donations or using the visitors' computer for cryptomining. But these models do not seem to be a real substitute for advertisement networks. This proposal presents a concept that could be a real substitute for the online advertisement business.

\section{Problem}
Advertisements can help to promote commercial products or change a general opinion. With commercial products, a company pays the publisher in order to let them show an advertisement to a visitor. The company needs to compensate for this payment, so in end effect the product will be sold with an additional charge compared to products without advertisements. This means that the end user is still paying indirectly to the publisher, which makes the whole ecosystem sub optimal from an end user point of view. So why not pay to the publisher directly?

At this moment there is no universal way of paying automatically to publishers that are serving advertisements right now.

\section{State of the art}
\label{stateoftheart}
There are basically two types of successful alternative business models for publishers to generate revenue: pay-per-view and a subscription. The problem with the first model is too much effort to make a small payment to a publisher when you just want to read one online article you stumbled upon. Blendle\footnote{\url{https://blendle.com}} tries to solve this by acting as a middleman where you are paying automatically a small amount when opening an article through their platform/ The prices range between €0,09 and €1,99 per article of which 30\% is paid to Blendle \footfullcite{blendledutchnewsplatform2014}. Despite the high fees, the platform is still successful with (\$ 3M in revenue), so it turns out that people are willing to pay for content.

The second model, subscription based, functions like the old fashioned newspaper subscriptions.  The problem with this approach is that the user is tied to one particular publisher and cannot switch to another one. Have multiple subscriptions would add up in costs quickly. An example of such a publisher is the Correspondent\footnote{\url{https://thecorrespondent.com}}, which is also really successful and expanded to the US a couple of months ago. One difference between a traditional subscription model is that it is based on pay what you wish.

\section {Concept}
As stated in paragraph \ref{stateoftheart}, users are willing to pay for content. Unfortunately, there is no universal way of paying for content, without loosing money to a third party or without much effort for the end user. Over the last decade, a lot of research has been done in the field of decentralized payment methods using blockchain technology. This concept tries to adapt this engineering science in order to create a universal way of paying for content automatically.

Browser plugins like metamask \footnote{\url{https://metamask.io}} enables a user to make payments directly from the browser, however this still requires the user to install a plugin.

The following structure should lead to a payment system that does not need any configuration. This is done by combining the crytography possibilities of javascript with the relatively new webRTC API, that is present in all modern browsers.

First the user visits the website of the publisher. The publisher has embedded a javascript library \textit{payjs-publisher.js} which scans the local machine for any open webRTC ports. If there is a wallet running, the library connects to it. If there is no wallet active, the library embeds an iframe with a hosted instance of \textit{payjs-wallet.js}. At this point, there is a connection with a sandboxed wallet.

Now the publisher can request money from wallet by sending his address. In order to prevent fraud, a maximum pay per hour per address is set by default in combination with a maximum of new addresses per hour. The addresses are also containing the domainname in hashed form, so that is is possible to blacklist certain publishers.


\newpage

\begin{figure}[htbp]
	\includesvg{images/structure.svg}
	\caption{Concept}
\end{figure}



\end{document}