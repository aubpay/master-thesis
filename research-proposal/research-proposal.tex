\documentclass[a4paper]{article}

\usepackage[backend=biber]{biblatex}
\usepackage{hyperref}
\usepackage{svg}

\renewcommand{\familydefault}{\sfdefault}

\bibliography{references}

\title{Research Proposal}
\author{Daan Middendorp}
\date{September 2019}

\begin{document}

\maketitle

\section{Introduction}

\sffamily
The business of online advertising has evolved into a landscape which is not transparent anymore. A handful of large advertisement firms are controlling practically every online ad you see. Almost every movement during the visit of a regular website is sent in an obfuscated way to the advertisement broker, without any visible sign to the visitor. This makes the whole browsing experience obnoxious, especially now it turns out that entire societies are being influenced by the power of advertisement networks, as we have seen in the Cambridge Analytica scandal \footfullcite{cadwalladr2018cambridge}.

Several publishers have been experimenting with alternative ways of generating income. Currently, some of them are selling subscriptions, asking for donations or using the visitors' computer for cryptomining \footfullcite{ruth2018digging}. But these models do not seem to be a real substitute for advertisement networks. This proposal presents a concept that could be a real substitute for the online advertisement business.

\section{Problem}
Advertisements can help to promote commercial products or change a general opinion. With commercial products, a company pays the publisher in order to let them show an advertisement to a visitor. The company needs to compensate for this payment, so in end effect the product will be sold with an additional charge compared to a product without advertisements. This means that the visitor is still paying indirectly to the publisher, which makes the whole ecosystem sub optimal from a visitor point of view. So why not pay to the publisher directly?

At this moment there is no \textbf{universal} way of paying automatically to a publisher, so that it is an alternative to advertisements.

\section{State of the art}
\label{stateoftheart}
In the current online publishing business, there are basically two types of relative successful alternative models to generate revenue: \textit{pay-per-view }and a \textit{subscription}. The problem with the first model is the amount of effort that it takes to make a small payment to a publisher when the visitor just wants to read one online article he or she stumbled upon. A company named Blendle\footnote{\url{https://blendle.com}} tries to solve this by acting as a middleman where you are paying automatically a small amount when opening an article through their platform. Prices range between €0,09 and €1,99 per article of which 30\% is being paid to Blendle \footfullcite{blendledutchnewsplatform2014}. Despite the high fees, the platform is still successful (\$ 3M in annual revenue), so it turns out that people are willing to pay for content.

The second model, subscription based, functions like the old-fashioned newspaper subscriptions. The problem with this approach is that the user is tied to one particular publisher and cannot switch to another one. Have multiple subscriptions would add up in costs quickly. An example of such a publisher is the Correspondent\footnote{\url{https://thecorrespondent.com}}, which is also really successful and expanded to the US a couple of months ago.

\section {Concept}
As stated in paragraph \ref{stateoftheart}, users are willing to pay for articles. Unfortunately, there is no universal way of paying automatically for content, without loosing money to a third party or asking a visitor to put a lot of effort in it. Over the last decade, the concept of blockchain technology has brought a lot of new possibilities into the field of digital payments. This concept tries to adapt this engineering science in order to create a universal way of paying for content automatically.

A browser plugins like Metamask\footnote{\url{https://metamask.io}} enables a user to make payments directly from the browser, however this still requires the user to install a plugin.

The following structure should lead to a payment system that does not need any configuration. This is done by combining the cryptography possibilities of javascript in combination with the relatively new webRTC API, which is available in all modern browsers.

\newpage
\subsection{Workflow}
The visitor opens an article on the website of the publisher. A javascript library \textit{payjs-publisher.js} is embedded into the response which scans the local machine for any running wallets using webRTC. If there is a wallet running on that machine, the library connects to it. If there is no wallet active, the library embeds an iframe with a hosted instance of \textit{payjs-wallet.js}. At this point, there is a connection with a sandboxed wallet.

Now the publisher can request money from that wallet by sending his address. In order to prevent fraud, a maximum pay per hour per address is set by default in combination with a maximum of new addresses per hour. The addresses are also containing the domain name in hashed form, so that it is possible to blacklist certain publishers.

The wallet transmits the transactions to the network, so that the publisher gets paid, even when no advertisements are shown.



\begin{figure}[htbp]
	\includesvg{images/structure.svg}
	\caption{Concept}
\end{figure}



\end{document}